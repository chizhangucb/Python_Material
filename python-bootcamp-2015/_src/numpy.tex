\chapter{NumPy, SciPy, and matplolib}

\begin{abstract}
Introduce Python's core numerical, scientific, and plotting packages.

\begin{itemize}
\item Fernando Pérez, Brian E. Granger, and John D. Hunter. "Python: an ecosystem for
scientific computing." \emph{Computing in Science \& Engineering} 13, no. 2 (2011):
13-21.
\item Stéfan van der Walt, S. Chris Colbert, and Gael Varoquaux. "The NumPy array: a
structure for efficient numerical computation." \emph{Computing in Science \&
Engineering} 13, no. 2 (2011): 22-30.
\item John D. Hunter. "Matplotlib: A 2D graphics environment." \emph{Computing
in Science \& Engineering} 9, no. 3 (2007): 0090-95.
\end{itemize}
\end{abstract}

\section{Introduction}
\begin{itemize}
\item \url{http://docs.scipy.org/doc/}
\item \url{http://matplotlib.org/gallery.html}
\item \url{https://scipy-lectures.github.io/}
\item \url{https://github.com/ipython/ipython/wiki/A-gallery-of-interesting-IPython-Notebooks}
\end{itemize}

\section{NumPy and matplotlib}

\subsection{Exercise: lock 'n load}
For this exercise please work through Stéfan van der Walt's NumPy
lock 'n load.

\begin{itemize}
\item \url{https://github.com/stefanv/teaching/tree/master/2008_numpy_load_n_load}
\end{itemize}

\subsection{ndarray}

\begin{itemize}
\item \url{http://scipy-lectures.github.io/intro/numpy/array_object.html}
\item \url{http://scipy-lectures.github.io/intro/numpy/operations.html}
\end{itemize}

\subsection{2D plotting}

\begin{itemize}
\item \url{http://scipy-lectures.github.io/intro/matplotlib/matplotlib.html}
\end{itemize}

\section{Example: random walk redux}

Recall the random walk simulation from 243.

%% begin.rcode p1, engine='python'
%% end.rcode

Here is a version implemented as a class.

%% begin.rcode p2, engine='python'
%% end.rcode

\subsection{Exercise: Sierpinski triangle}
Write a Python script to construct Sierpinski triangle using the following
algorithm:

\begin{enumerate}
\item Choose 3 points in the plane (forming a triangle).
\item Choose another "starting" pointing (current position).
\item Randomly choose one of the corners of the triangle.
\item Move halfway from your current position to the selected corner.
\item Plot the new current position.
\item Repeat from step 3 (for 100 times).
\end{enumerate}

\section{SciPy}

\begin{itemize}
\item \url{http://scipy-lectures.github.io/intro/scipy.html}
\end{itemize}

\section{Exercise: State of the Union addresses}

For this exercise, you will revisit the State of the Union Addresses.
Load the data, use whatever tools and analysis you want, and use matplotlib
to make plots.
